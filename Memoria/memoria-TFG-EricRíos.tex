% ---------------------------------------------------
% Trabajo Fin de Grado
% Author: Eric Ríos Hamilton <alu0101549835@ull.edu.es>
% Chapter: Preface 
% ----------------------------------------------------
 
\documentclass[english,a4paper,12pt,oneside]{extreport}
%\documentclass[a4paper, twoside, 12pt]{book}
\usepackage{listingsutf8}
\usepackage{listings}
\usepackage[scale=0.82]{FiraMono}
\usepackage[T1]{fontenc}
\usepackage[a4paper]{geometry}
\usepackage[spanish, main=english]{babel}
\usepackage[utf8]{inputenc}
\usepackage{lmodern,textcomp}  % For € symbol
\usepackage{pdflscape}
\usepackage{amsmath}
\usepackage{array}
\newcolumntype{C}[1]{>{\centering\arraybackslash}m{#1}}
\usepackage{makecell}
\usepackage{multirow}
\usepackage{ccicons}
\usepackage{float}
\floatstyle{ruled}
\newfloat{Code}{htbp}{lop}
\floatname{Code}{Listing}
%%%%%%%%%%%%%%%%%%%%%%%%%%%%%%%%%%%%%%%%%%%%%%%%%%%%%%%%%%%%%%%%%%%%%%%%%%%%%%%%%%%%%%%%%%%%
% Next 3+3 lines select PDF or PS output (comment as apropriate)
% To switch from PDF and PS comment/uncomment here and change Makefile
\usepackage[pdftex]{xcolor}
\usepackage[pdftex]{graphicx}
\graphicspath{{images/}}
%\usepackage[dvips]{color}
%\usepackage[dvips]{graphicx}
\usepackage{epsfig}
%\graphicspath{{images/eps/}}
\usepackage{floatrow}
\usepackage{tabu}
\usepackage{tabularx}
%%%%%%%%%%%%%%%%%%%%%%%%%%%%%%%%%%%%%%%%%%%%%%%%%%%%%%%%%%%%%%%%%%%%%%%%%%%%%%%%%%%%%%%%%%%%
\usepackage{algorithmic}
%\usepackage[pdftex=true,colorlinks=false,urlcolor=blue,plainpages=false,pagebackref=true,citecolor=red]{hyperref} %hiperenlaces y backcites
\usepackage[colorlinks=true,urlcolor=blue,plainpages=false,pagebackref=true,citecolor=black]{hyperref} %hiperenlaces y backcites
\hypersetup{breaklinks=true}
% Redefinir el comando \url para cambiar el tamaño del texto de los enlaces
% \let\oldurl\url
% \renewcommand{\url}[1]{{\scriptsize{\oldurl{#1}}}}

\usepackage{url}
\usepackage{subcaption}
%%%%%%%%%%%%%%%%%%%%%%%%%%%%%%%%%%%%%%%%%%%%%%%%%%%%%%%%%%%%%%%%%%%%%%%%%%%%%%%%%%%%%%%%%%%
% Comandos para escribir "siempre igual"
\newcommand{\TitleTopic}{\texttt{Automatic Procedural Terrain Generation for Games}}


%%% Traducimos el pseudocodigo
\renewcommand{\algorithmicwhile}{\textbf{mientras}}

%%%%%%%%%%%%%%%%% Se crea un entorno para listar código fuente %%%%%%%%%%%%%%%
\newenvironment{sourcecode}
{\begin{list}{}{\setlength{\leftmargin}{1em}}\item\scriptsize\bfseries}
{\end{list}}

\newenvironment{littlesourcecode}
{\begin{list}{}{\setlength{\leftmargin}{1em}}\item\tiny\bfseries}
{\end{list}}

\newenvironment{summary}
{\par\noindent\begin{center}\textbf{Abstract}\end{center}\begin{itshape}\par\noindent}
{\end{itshape}}

\newenvironment{keywords}
{\begin{list}{}{\setlength{\leftmargin}{1em}}\item[\hskip\labelsep \bfseries Keywords:]}
{\end{list}}

\newenvironment{palabrasClave}
{\begin{list}{}{\setlength{\leftmargin}{1em}}\item[\hskip\labelsep \bfseries Palabras clave:]}
{\end{list}}

%%%%%%%%%%%%%%%%%%%%%%%%%%%%%%%%%%%%%%%%%%%%%%%%%%%%%%%%%%%%%%%%%%%%%%%%%%%%%%%
\definecolor{marron}       {rgb}{0.496, 0.203, 0.152}
\definecolor{verde-claro}  {rgb}{0.625, 0.734, 0.199}
\definecolor{oscuro}       {rgb}{0.187, 0.141, 0.285}
\definecolor{gris}     	   {rgb}{0.500, 0.500, 0.500}
\definecolor{bgd-listings} {rgb}{0.999, 0.999, 0.900}
\definecolor{gray97}{gray}{.97}
\definecolor{gray75}{gray}{.75}
\definecolor{gray45}{gray}{.45}
\definecolor{gray}{gray}{.45}
\definecolor{Brown}{cmyk}{0,0.81,1,0.60}
\definecolor{OliveGreen}{cmyk}{0.64,0,0.95,0.40}
\definecolor{CadetBlue}{cmyk}{0.62,0.57,0.23,0}
\definecolor{lightlightgray}{gray}{0.9}
%%%%%%%%%%%%%%%%%%%%%%%%%%%%%%%%%%%%%%%%%%%%%%%%%%%%%%%%%%%%%%%%%%%%%%%%%%%%%%%%%%%%%%%%%%%
%Evitar partir palabras al final de la línea
\hyphenpenalty=10000
\tolerance=1000
%%%%%%%%%%%%%%%%%%%%%%%%%%%%%%%%%%%%%%%%%%%%%%%%%%%%%%%%%%%%%%%%%%%%%%%%%%%%%%%%%%%%%%%%%%%%
% Para listados de código
\usepackage{listings}
% \lstloadlanguages{C,C++}
%\lstloadlanguages{python,C}

% Definiendo colores para los listados de código fuente - Univ. Deusto
\definecolor{violet}{rgb}{0.5,0,0.5}
\definecolor{lightgray}{rgb}{.9,.9,.9}
\definecolor{darkgray}{rgb}{.4,.4,.4}
\definecolor{purple}{rgb}{0.65, 0.12, 0.82}
\definecolor{navy}{rgb}{0,0,0.5}
\definecolor{hellgelb}{rgb}{1,1,0.8}
\definecolor{colKeys}{rgb}{0,0,1}
\definecolor{colIdentifier}{rgb}{0,0,0}
\definecolor{colComments}{rgb}{1,0,0}
\definecolor{colString}{rgb}{0,0.5,0}
\definecolor{main-color}{rgb}{0,0,0}
\definecolor{back-color}{rgb}{0.1686, 0.1686, 0.1686}
\definecolor{string-color}{HTML}{953800}
\definecolor{key-color}{HTML}{8250DF}
\definecolor{comment-color}{HTML}{008000}
\definecolor{sycl-color}{HTML}{CF222E}
\definecolor{highlight-color}{HTML}{953800}

%\lstset{morekeywords={pragma copy\_in copy\_out copy omp parallel private reduction shared hicuda loop\_partition over\_tblock over\_thread}}
\lstdefinelanguage{C++}{
  keywords={new, true, false, catch, try, return, null, switch, const, if, in, while, do, else, case, break},
  ndkeywords={class, throw, this},
  sensitive=false,
  comment=[l]{//},
  morecomment=[s]{/*}{*/},
  morestring=[b]',
  morestring=[b]"
}

\lstdefinestyle{cppstyle}{
    language=C++,
    basicstyle = {\ttfamily\color{main-color}\footnotesize},
    backgroundcolor = {\color{back-color}},
    stringstyle = {\color{string-color}},
    keywordstyle = {\color{key-color}},
    keywordstyle = [2]{\color{gray}},
    keywordstyle = [3]{\color{sycl-color}},
    keywordstyle = [4]{\color{highlight-color}},
    keywordstyle = [5]{\color{black}},
    commentstyle = {\color{comment-color}},
    otherkeywords = {++,--,<<,>>,<<<,>>>,+,-,\{,\},\[,\],*,\&,sycl,std},
    morekeywords = [2]{++,--,<<,>>},
    morekeywords = [3]{sycl},
    morekeywords = [4]{std,+,-,{,},[,],*,\&},
    morekeywords = [5]{<<<,>>>},
    columns=flexible,
    tabsize=2,
    frame=single,
    extendedchars=true,
    showspaces=false,
    showstringspaces=false,
    numbers=left,
    numberstyle=\tiny,
    breaklines=true,
    backgroundcolor=\color{white},
    breakautoindent=true,
    captionpos=b
}

\lstdefinelanguage{YAML}{
  morekeywords={true,false,null,y,n},
  sensitive=false,
  morecomment=[l]{\#},
  morestring=[b]",
  morestring=[b]'
}

\lstdefinestyle{yamlstyle}{
    language=YAML,
    basicstyle = {\ttfamily\color{main-color}\footnotesize},
    backgroundcolor = {\color{lightgray}},
    stringstyle = {\color{string-color}},
    keywordstyle = {\color{key-color}},
    commentstyle = {\color{comment-color}},
    otherkeywords = {:, -},  % YAML punctuation
    columns=flexible,
    tabsize=2,
    frame=single,
    extendedchars=true,
    showspaces=false,
    showstringspaces=false,
    numbers=left,
    numberstyle=\tiny,
    breaklines=true,
    breakautoindent=true,
    captionpos=b
}

\lstdefinelanguage{PythonCustom}{
  morekeywords={
    def, class, return, if, else, elif, try, except, raise, while, for, in, import, from, as, pass, break, continue, True, False, None, with, lambda, yield, global, nonlocal
  },
  sensitive=true,
  morecomment=[l]{\#},
  morestring=[b]',
  morestring=[b]"
}

\lstdefinestyle{pythonstyle}{
    language=PythonCustom,
    basicstyle = {\ttfamily\color{main-color}\footnotesize},
    backgroundcolor = {\color{lightgray}},
    stringstyle = {\color{string-color}},
    keywordstyle = {\color{key-color}},
    commentstyle = {\color{comment-color}},
    otherkeywords = {:, (, ), [, ], \{, \}, =, +, -, *, /, <, >},
    columns=flexible,
    tabsize=2,
    frame=single,
    extendedchars=true,
    showspaces=false,
    showstringspaces=false,
    numbers=left,
    numberstyle=\tiny,
    breaklines=true,
    breakautoindent=true,
    captionpos=b
}

\lstset{literate=%
   *{0}{{{\color{darkgray}0}}}1
    {1}{{{\color{darkgray}1}}}1
    {2}{{{\color{darkgray}2}}}1
    {3}{{{\color{darkgray}3}}}1
    {4}{{{\color{darkgray}4}}}1
    {5}{{{\color{darkgray}5}}}1
    {6}{{{\color{darkgray}6}}}1
    {7}{{{\color{darkgray}7}}}1
    {8}{{{\color{darkgray}8}}}1
    {9}{{{\color{darkgray}9}}}1
} 

\lstdefinelanguage{log}{
  keywords={},
  sensitive=false
}

\lstdefinestyle{logstyle}{
    language=log,
    basicstyle = {\ttfamily\color{main-color}\scriptsize},
    backgroundcolor = {\color{back-color}},
    columns=flexible,
    tabsize=2,
    frame=single,
    extendedchars=true,
    showspaces=false,
    showstringspaces=false,
    numbers=left,
    numberstyle=\tiny,
    breaklines=true,
    backgroundcolor=\color{white},
    breakautoindent=true,
    captionpos=b
}

% Otro formato más bonito para código fuente
\newcommand{\codigofuente}[3]{%
  \lstlisting[language=#1,caption={#2}]{#3}%
}
%%%%%%%%%%%%%%%%%%%%%%%%%%%%%%%%%%%%%%%%%%%%%%%%%%%%%%%%%%%%%%%%%%%%%%%%%%%%%%%
\begin{document}
%% Para los metadatos del fichero PDF que se genera:
\hypersetup{
    pdftitle={Automatic Procedural Terrain Generation for Games},
    pdfauthor={Eric Ríos Hamilton},
    pdfkeywords={Terrain Generation},
    pdfsubject={Final Degree Project on Procedural Terrain Generation for Games}
}

\renewcommand{\lstlistingname}{Listing}% Listing -> Listado de código

\pagestyle{empty}
\newcommand{\HRule}{\rule{\linewidth}{0.3mm}}
{
    \setlength{\parindent}{0mm}
    \setlength{\parskip}{0mm}
    
    \vspace*{1.20cm}
    \includegraphics[width=9.81cm]{Memoria/images/escuela-ingenieria-tecnologia-original}
    \vspace*{\stretch{0.9}}
    
    {\centering
    \fontsize{32pt}{32pt}\selectfont Trabajo de Fin de Grado\\[10pt]
    \fontsize{20pt}{20pt}\selectfont Grado en Ingeniería Informática\par}
    \HRule\vspace*{-2mm}
    \begin{flushright}
        {\fontsize{32pt}{32pt}\selectfont Automatic Procedural Terrain Generation for Games\par
        \vspace*{3mm}
        \fontsize{18pt}{18pt}\selectfont \textit{Generación procedural automatizada de terreno para juegos}\par
        \vspace*{11mm}
        \fontsize{16pt}{16pt}\selectfont Eric Ríos Hamilton
        }\vspace*{12mm}
    \end{flushright}
    \HRule
    
    \vspace*{\stretch{1}}
    \begin{center}
        \fontsize{18pt}{18pt}\selectfont La Laguna, \today
    \end{center}
}

%%%%%%%%%%%%%%%%%%%%%%%%%%%%%%%%%%%%%%%%%%%%%%%%%%%%%%%%%%%%%%%%%%%%%%%%%%%%%%%
% Signature page (add the official stamp)
%%%%%%%%%%%%%%%%%%%%%%%%%%%%%%%%%%%%%%%%%%%%%%%%%%%%%%%%%%%%%%%%%%%%%%%%%%%%%%%
\newpage
%\cleardoublepage
\thispagestyle{empty}

D. {\bf Francisco de Sande González}, profesor Titular de Universidad adscrito al Departamento de Ingeniería Informática y de Sistemas de la Universidad de La Laguna, en calidad de tutor 

\bigskip

\bigskip
\bigskip
{\bf C E R T I F I C A}

\bigskip
\bigskip
\bigskip
Que el presente trabajo de Fin de Grado titulado:

\bigskip
``{\it \TitleTopic{}}''

\bigskip
\bigskip
\bigskip
%Cambiar
\noindent ha sido realizado bajo su dirección por D. {\bf Eric Ríos Hamilton}.

\bigskip
\bigskip

Y para que así conste, en cumplimiento de la legislación vigente y a los efectos
oportunos firman la presente memoria del Trabajo en La Laguna a {\selectlanguage{spanish}\today}.

%\cleardoublepage
\newpage
%%%%%%%%%%%%%%%%%%%%%%%%%%%%%%%%%%%%%%%%%%%%%%%%%%%%%%%%%%%%%%%%%%%%%%%%%%%%%%%
\thispagestyle{empty}

{ \flushright

\begin{LARGE}
Agradecimientos
\end{LARGE}

\hspace{3mm}

\begin{large}


\hspace{3mm}

\hspace{3mm}
Agradecimientos 

Este trabajo ha sido financiado por el Ministerio de Ciencia e Innovación a través de los proyectos 
PID2023-151073NB-I00, 
TED2021-131019B-I00 y
PDC2022-134013-I00 
\end{large}

}

%%%%%%%%%%%%%%%%%%%%%%%%%%%%%%%%%%%%%%%%%%%%%%%%%%%%%%%%%%%%%%%%%%%%%%%%%%%%%%%%%
\newpage

\begin{huge}
Licencia
\end{huge}

\bigskip
%* Si quiere permitir que se compartan las adaptaciones de tu obra mientras se comparta de la misma manera
%y NO quieres permitir usos comerciales de tu obra indica:

\begin{center}
  \includegraphics[height=20mm]{images/by-nc-sa.eu} \\[1ex]
  {\large © Esta obra está bajo una licencia de Creative Commons Reconocimiento-NoComercial-CompartirIgual 4.0 Internacional.}
\end{center}

%%%%%%%%%%%%%%%%%%%%%%%%%%%%%%%%%%%%%%%%%%%%%%%%%%%%%%%%%%%%%%%%%%%%%%%%%%%%%%%
\newpage  %\cleardoublepage
\selectlanguage{spanish}
\begin{abstract}
{\em
El objetivo de este trabajo ha sido ...
}

\begin{palabrasClave}
Aprendizaje Profundo, Imágenes Multiespectrales, Sentinel-2, Segmentación Semántica, Detección de Carreteras, Redes Neuronales Convolucionales, OpenStreetMap, Teledetección.
\end{palabrasClave}
\end{abstract}
\selectlanguage{english}
%%%%%%%%%%%%%%%%%%%%%%%%%%%%%%%%%%%%%%%%%%%%%%%%%%%%%%%%%%%%%%%%%%%%%%%%%%%%%%%
\newpage  %\cleardoublepage
\begin{abstract}
{\em
The objective of this work has been ...
}

\begin{keywords}
 Deep Learning, Multispectral Images, Sentinel-2, Semantic Segmentation, Road Detection, Convolutional Neural Networks, OpenStreetMap, Remote Sensing.
\end{keywords}

\end{abstract}
%%%%%%%%%%%%%%%%%%%%%%%%%%%%%%%%%%%%%%%%%%%%%%%%%%%%%%%%%%%%%%%%%%%%%%%%%%%%%%%
\newpage{\pagestyle{empty}}
\thispagestyle{empty}


\pagestyle{myheadings} %my head defined by markboth or markright
% No funciona bien \markboth sin "twoside" en \documentclass, pero al
% ponerlo se dan un montón de errores de underfull \vbox, con lo que no se
% ha puesto.
\markboth{Eric Ríos Hamilton}{ULL}

%%%%%%%%%%%%%%%%%%%%%%%%%%%%%%%%%%%%%%%%%%%%%%%%%%%%%%%%%%%%%%%%%%%%%%%%%%%%%%%
%Numeracion en romanos
\renewcommand{\thepage}{\roman{page}}
\setcounter{page}{1}

%%%%%%%%%%%%%%%%%%%%%%%%%%%%%%%%%%%%%%%%%%%%%%%%%%%%%%%%%%%%%%%%%%%%%%%%%%%%%%%
{
\hypersetup{linkcolor=black}
\tableofcontents
}
%%%%%%%%%%%%%%%%%%%%%%%%%%%%%%%%%%%%%%%%%%%%%%%%%%%%%%%%%%%%%%%%%%%%%%%%%%%%%%%
\newpage{\pagestyle{empty}}
{
\hypersetup{linkcolor=black}
\listoffigures
}
%%%%%%%%%%%%%%%%%%%%%%%%%%%%%%%%%%%%%%%%%%%%%%%%%%%%%%%%%%%%%%%%%%%%%%%%%%%%%%%
\newpage{\pagestyle{empty}}
{
\hypersetup{linkcolor=black}
\listoftables
}
%%%%%%%%%%%%%%%%%%%%%%%%%%%%%%%%%%%%%%%%%%%%%%%%%%%%%%%%%%%%%%%%%%%%%%%%%%%%%%%
\newpage{\pagestyle{empty}}
%%%%%%%%%%%%%%%%%%%%%%%%%%%%%%%%%%%%%%%%%%%%%%%%%%%%%%%%%%%%%%%%%%%%%%%%%%%%%%%
%Numeracion a partir del capitulo I
\renewcommand{\thepage}{\arabic{page}}
\setcounter{page}{1}
% ==========================================================
% --------               Capítulos                ----------
% --------    Estan en el directorio capitulos/   ----------
% ==========================================================
% Comentar la siguiente línea en la versión final
% ---------------------------------------------------
% Trabajo Final de Grado
% Author: Eric Ríos Hamilton <alu0101549835@ull.edu.es>
% Chapter: Instrucciones y consejos para la redacción de la memoria
% ----------------------------------------------------


\chapter*{Instrucciones y consejos para la redacción de la memoria}
La inclusión de este capítulo deberá comentarse (para su eliminación) en la versión definitiva del documento.

Se describen a continuación algunas pautas que debieran seguirse a la hora de redactar el documento

\begin{enumerate}
\item En Latex solo dejar una línea en blanco después de un punto y aparte produce un punto y aparte en el renderizado.
Si los párrafos se colocan en líneas diferentes (como se hace en el código latex de este párrafo) ello NO genera un punto y aparte en el pdf renderizado.
Escribir SIEMPRE un retorno de carro duro después de cada punto.
Dejar una línea en blanco a continuación, si se quiere un punto y aparte en el PDF resultante.

\item Utilizar comentarios de OverLeaf para indicar cuestiones que queden pendientes en la redacción del documento o que requieren alguna revisión.
En la versión final no debiera quedar ninguno de estos comentarios.

\item Ejemplo de cómo realizar una referencia a una cita bibliográfica: Una vía que algunos economistas \cite{Smythe:2022:GMS} han comenzado a utilizar para realizar los conocidos como mapas de pobreza consiste en la aplicación de algoritmos de aprendizaje automático a imágenes satelitales.

\item Revisar el fichero \texttt{bibliography.bib} del proyecto. Contiene todas las referencias bibliográficas en formato BibTex.
Ver en ese fichero diferentes tipos de referencias: libros, artículos, proceedings de congresos, etc.
El fichero .bib del proyecto puede contener muchas entradas, pero solo se utilizan aquellas que se citan en el código latex.

\item La bibliografía debería incluirse en el formato especificado y para cada uno de los ítems de la bibliografía debería haber una cita a ese item en el texto del documento: no puede haber una referencia a la que no se mencione en el texto. Si se pone una referencia, habrá que buscar una razón para que figure en la bibliografía.

\item Las ''claves' para las referencias bibliográficas deben tener (en el fichero \texttt{.bib} del proyecto) un formato similar al siguiente: \texttt{Smythe:2022:GMS}
Donde Smythe es el apellido del primer autor, 2022 es el año de la publicación y GMS son las iniciales de las tres primeras palabras significativas del título (en este caso es \textit{Geographic microtargeting of social assistance with high-resolution poverty maps}, y de ahí GMS.

\item En la Bibliografía debe figurar ineludiblemente una referencia \cite{URL::TFG} al repositorio de código del trabajo. Ese repositorio debería hacerse público y hay que recordar entregarlo a todos los miembros del tribunal, junto con la memoria.

\item Las referencias que no sean muy importantes y que sean URLs que se hayan consultado, o correspondientes a productos o tecnologías no publicados en artículos o revistas, es preferible incluirlos como notas a pie de página. Va un ejemplo: 

The labor cost is calculated at \textbf{20.00€/hour}, based on the average salary of a junior deep learning developer in Spain (35,000€-37,000€ annually), sourced from PayScale\footnote{https://www.payscale.com/research/ES/Job=Machine\_Learning\_Engineer/Salary}, Glassdoor España\footnote{https://www.glassdoor.es/Sueldos/machine-learning-engineer-sueldo-SRCH\_KO0,25.htm}, and specialized tech salary reports from Manfred\footnote{https://www.getmanfred.com/en/blog/guia-salarial-2024-salarios-en-tecnologia-espana-manfred}.

\item Las palabras \textit{Figura}, \textit{Tabla} y \textit{Listado} se escribirán siempre con mayúscula inicial.

\item Se intentará que los listados de código (Listings) no se corten entre dos páginas. 
Ese formateo es aconsejable dejarlo para el final de la redacción, puesto que los elementos (figuras, tablas, etc.) se mueven de su posición conforme se añade o elimina texto.

\item Se intentará que en los listados de código (Listings) haya referencias al repositorio de código en el que el lector pueda ver el código completo que se expone (a veces de forma resumida) en el listado.

\item Una vez que se introduzca un acrónimo (ULL, por ejemplo) utilizarlo siempre de ahí en adelante, salvo que se pretenda dar un énfasis concreto a la frase, en cuyo caso volveremos a usar el texto correspondiente al acrónmio (Universidad de La Laguna).

\item Aparce el Listing X pero no hay mención alguna a lo que se expone en el mismo. Eso no debería ocurrir para ningún Listing, Figura o Tabla

\item En Latex NO puedes confiar en que un listado esté en una determinada posición del texto: los listings, figuras y tablas "flotan" por el texto. Por eso te tienes que referir a ellos por su número y tienes que explicar lo que quiera que sea haciendo referencia a ellos.

\item Todos los captions de los Listings han de tener un hipervínculo al correspondiente código que el lector podrá ver en el repo GitHub del proyecto. 

\item Cada vez que se incluye una figura (Listing, Tabla, etc.) en el texto, HAY QUE DECIR algo sobre ella, comentarla.
No puede ser que uno de estos elementos aparezca en el texto y no se comente.

A modo de ejemplo:

Figure \ref{fig:efficiency_analysis_quadro} presents the efficiency analysis executed on a Quadro M4000 GPU, revealing bla, bla. The left panel shows bla bla, while the right panel analyzes bla bla.

\item Listing \ref{listing:hpl-saxpy-embedded} displays how to perform a SAXPY (single precision A X plus Y) operation using the HPL embedded language.

Obsérvese que el listado incluye un hipervínculo a un repositorio donde el lector puede ver el código completo.

\item The training configuration is structured in three main components as shown in Listing \ref{listing:training_config}.
Comments in the Listing \ref{listing:training_config} code explain the meaning of some of these parameters.

Este ejemplo es una alternativa para incluir listados de código pero es preferible el anterior: separar los listados en ficheros independientes separados del código latex de la memoria.

\item Table \ref{tab:sentinel2_bands_detailed} presents an example of Table to use in the text.

\item Al finalizar la redacción del documento, buscar en toda el PDF generado la cadena \texttt{?}. 
Corresponde con referencias (a capítulos, tablas, etc.) que por alguna razón no se han renderizado bien al compilar el Latex.

\item También al finalizar la redacción, identificar todos los listados de cópdigo (Listing), Tablas y Figuras y asegurar de que los listados de código no quedan partidos entre páginas y que tablas y figuras no generan espacio adicional en el PDF una vez renderizado.

\item La memoria del TFG se entrega a través de un procedimiento en sede electrónica de la ULL a través del cual se solicita la firma del tutor del TFG.
\end{enumerate}




%% ===================================================
\begin{figure}[H]
    \centering
    \includegraphics[width=1.0\textwidth]{Memoria/images/efficiency_analysis_quadro.png}
    \caption{Ejemplo de inclusión de una figura}
    \label{fig:efficiency_analysis_quadro}
\end{figure}
%% ===================================================

\subsubsection{Energy Efficiency Analysis}
For deployment scenarios, energy efficiency becomes a critical factor. Figure \ref{fig:energy_training} presents power consumption analysis for one training.

%% ===================================================
\begin{figure}[H]
\centering
\includegraphics[width=\textwidth]{Memoria/images/energy_consumption_training.png}
\caption{Otro ejemplo de figura. Total energy consumption for training in Joules (J) across GPU architectures.}
\label{fig:energy_training}
\end{figure}
%% ===================================================

%% ===================================================
\lstinputlisting[language=C++,style=cppstyle,caption={Ejemplo de listado de código. SAXPY on HPL (HPL embedded language). \href{{https://github.com/fraguela/hpl?tab=readme-ov-file}}{\textit{Original source}}.},label={listing:hpl-saxpy-embedded}]{listings/hpl_saxpy_embedded.cc}
%% ===================================================

%% ===================================================
\begin{lstlisting}[style=yamlstyle,
   caption={Este fichero NO está en un fichero separado del latex. \href{https://github.com/fsande/TFG-Alvaro-Fontenla/blob/main/Código/sentinel_paper/config_sentinel2.yaml}{\textit{See on GitHub}}.},label={listing:training_config}]
# Core Training Parameters
training:
  epochs: 300
  batch_size: 4
  learning_rate: 0.001
  weight_decay: 0.0001
  num_workers: 2             # Number of workers for data loading (CPU threads)
# Class Balancing Strategy
class_weights:
  road_weight: 8.0           # Addresses severe class imbalance in road segmentation
# Sentinel-2 Specific Configuration
sentinel2:
  target_resolution: 10      # meters
  osm_buffer_meters: 5       # buffer for road extraction
  min_road_pixels: 10        # minimum road pixels per crop
  crop_size: 512             # spatial dimensions
  max_crops_per_scene: 50    # prevents overfitting by limiting the number of crops per scene
\end{lstlisting}
%% ===================================================




%% ===================================================
\begin{table}[H]
\centering
\caption{Detailed characteristics of the Sentinel-2 bands used in this study.}
\label{tab:sentinel2_bands_detailed}
\renewcommand{\arraystretch}{1.3}
\begin{tabular}{|c|l|c|c|p{5.5cm}|}
\hline
\textbf{Band} & \textbf{Name} & \textbf{Wavelength} & \textbf{Resolution} & \textbf{Primary} \\
             &               & \textbf{(nm)}       & \textbf{(m)}        & \textbf{Applications} \\
\hline
B02 & Blue         & 490  & 10 & RGB visualization, atmospheric correction. \\
\hline
B03 & Green        & 560  & 10 & RGB visualization, vegetation health assessment. \\
\hline
B04 & Red          & 665  & 10 & RGB visualization, chlorophyll absorption. \\
\hline
B05 & Red Edge 1   & 705  & 20 & Vegetation stress detection. \\
\hline
B06 & Red Edge 2   & 740  & 20 & Leaf area index estimation. \\
\hline
B07 & Red Edge 3   & 783  & 20 & Vegetation classification. \\
\hline
B08 & NIR          & 842  & 10 & Biomass estimation, water body detection. \\
\hline
B8A & NIR Narrow   & 865  & 20 & Precise vegetation analysis. \\
\hline
B11 & SWIR 1       & 1610 & 20 & Moisture content, cloud detection. \\
\hline
B12 & SWIR 2       & 2190 & 20 & Geological mapping, cloud screening. \\
\hline
\end{tabular}
\end{table}
%% ===================================================

% ---------------------------------------------------
% Trabajo Final de Grado
% Author: Eric Ríos Hamilton <alu0101549835@ull.edu.es>
% Chapter: Preface 
% ----------------------------------------------------
\chapter*{Preface}
\addcontentsline{toc}{chapter}{Introduction} 
Este capítulo Preface, puede omitirse si se considera oportuno hacerlo, pero posiblemente lo que se dice a continuación, debería mantenerse en el documento final.

All the code examples written by the student for this work follow the principles of Martin's Clean Code book \cite{Martin:2009:Clean} and conform to the Google's Style Guide\footnote{\href{https://google.github.io/styleguide/}{{Google Style Guides} \url{https://google.github.io/styleguide/}}}, while code excerpts from other works will remain essentially untouched.
The reason behind the decision to write code using Google's Style is completely arbitrary, the importance of choosing a style relies on being consistent with its use: \textit{``The last thing we want to do is add more complexity to the source code by writing it in a jumble of different individual styles.'' - Robert C. Martin}.

Every piece of code written by the student for this project is publicly available on the GitHub repository dedicated to this work \cite{URL::TFG}.
Also note that every listing has a link to its original source and in the case of the student's code, there will be a direct link to the corresponding file in the work's repository, labeled as \textit{``See on GitHub''}.

% ---------------------------------------------------
% Trabajo Final de Grado
% Author: Eric Ríos Hamilton <alu0101549835@ull.edu.es>
% Chapter: Goals 
% ----------------------------------------------------


\chapter{Goals} \label{chap:Goals}
Este capítulo se escribe normalmente al final de la realización del trabajo. 
En ese momento se tendrá una perspectiva más clara de lo que se ha pretendido hacer y de lo que realmente se ha hecho.

This document summarizes the research and development work carried out by the student in the achievement of his Final Degree Project (\textit{Trabajo de Fin de Grado}, TFG), which will conclude his studies for the degree \textit{Grado en Ingeniería Informática} at the \textit{Escuela Superior de Ingeniería y Tecnología} at the University of La Laguna (ULL).

This project has the following main goals:

\begin{enumerate}
    \item A first objective has been ... 
    \item Another key objective is ... 
    \item The implementation and ...
    \item Additionally, the student will apply....
    \item Finally, after thorough experimentation and evaluation, the student is expected to deliver...
\end{enumerate}

% ---------------------------------------------------
% Trabajo Final de Grado
% Author: Eric Ríos Hamilton <alu0101549835@ull.edu.es>
% Chapter: Introduction 
% ----------------------------------------------------

\cleardoublepage
\chapter{Introduction} \label{chap:Introduction}
% Introducción al lector en el problema que se propone resolver.
% Procedural Content Generation \cite{Maleki:2024:PCG} (PCG) is defined as the automatic creation of game content using algorithms. 
% PCG has a long history in both the game industry and the academic world. 
% It can increase player engagement and ease the work of game designers. 
% While recent advances in deep learning approaches in PCG have enabled researchers and practitioners to create more sophisticated content, it is the arrival of Large Language Models (LLMs) that truly disrupted the trajectory of PCG advancement. 

The virtual worlds explorable in video games are at the core of the player experience, interacting with and influencing every other characteristic of the work. 
From gameplay mechanics to graphics, and everything in between, each element of a video game must work in consonance with its virtual representation of space.
As games grow more ambitious, a need of huge, organic worlds for the player to explore and get lost in arises.
However, as the size their worlds increases, so does the effort and man-power required to hand-craft them.
To avoid impossible workloads and streamline development processes many studios have adopted procedural generation to algorithmically generate their game worlds.

\section{Tools \& Technologies}
\begin{itemize}
    \item{\textbf{Engine}} 
    Godot\footnote{https://godotengine.org/} will be used as a graphics and physics engine.
    It is a free, open source, lightweight engine that can easily be modified to suit various purposes and supports a wide array of programming languages.
    Its ease of use and extensibility makes it ideal for this kind of project.
    \item{\textbf{Languages}} 
    \begin{itemize}
        %C# - Lenguaje con el que más familiarizado estoy. Suficientemente eficiente, muchísimo más cómodo y rápido para desarrollar que C++.
        \item{\textbf{GDScript}}\footnote{https://gdscript.com/} 
        is an interpreted scripting, object-oriented language designed for the ground up for the Godot engine.
        It is great for quick prototyping and iteration, although it not always sports the best performance. 
        In situations where the best performance is critical, other languages will be taken into account.
        \item{\textbf{C++}}\footnote{https://isocpp.org/} 
        is a proven multiparadigm language.
        It will mainly be used for its performance benefits over GDScript, especially in parallelization.
        \item{\textbf{The OpenGL Shading Language (GLSL)}}\footnote{https://registry.khronos.org/OpenGL/index\_gl.php}
        is a high-level shading language with a syntax similar to that of C.
        It provides relatively simple access to the GPU, allowing for the execution of complex algorithms in it.
        This language will be used to design compute shaders when the extreme parallelization capabilities of the GPU suit a solution to a problem.
        \item{\textbf{Godot Shading Language (GDSL)}} 
        is a shading language designed for the Godot engine.
        It is based on GLSL, but provides some useful features for integration within the engine.
        It will be used for visual shaders.
        \end{itemize}
    \item{\textbf{Integrated development environment (IDE)}}
    JetBrains Rider \footnote{https://www.jetbrains.com/rider/} is a .NET and game development centered IDE.
    Although originally .NET focused, its widespread use in game development has driven it to adopt other game development related technologies, such as Godot integration.
    It is free of use for non-commercial projects and possesses ample capabilities, such as integrated debugging and dynamic program analysis.
    \item{\textbf{Version control}} GitHub \footnote{https://github.com/} is a development platform that provides Git\footnote{https://git-scm.com/}-based distributed version control.
    It will be used to host the repository worked on in this project.
    \item{\textbf{Programming assistance}}
    GitHub Copilot \footnote{https://github.com/features/copilot} is an AI assistant integrated into the IDE. 
    It will be used for coding assistance during development.
\end{itemize}
% ---------------------------------------------------
% Trabajo Final de Grado
% Author: Eric Ríos Hamilton <alu0101549835@ull.edu.es>
% Chapter: Related Work
% ----------------------------------------------------

\cleardoublepage
\chapter{Related Work} \label{chap:Related_Work}
% Reseñar/Revisar aquí algunos de los trabajos más relevantes (related work) que hacen cosas similares a las que nosotros queremos hacer.

% Algunos de estos trabajos parecen relevantes.
% En cualquier caso, tener en cuenta la fecha de publicación de cada uno de ellos.
% Los más recientes sería conveniente revisarlos y a partir de ellos hallar otros trabajos relacionados.

\section{Content Generation}
Procedural content generation (PCG) is a common way of tackling the aforementioned issue of unapproachable world sizes and amounts of assets and can be defined as \textit{the automatic creation of digital assets for games,
simulations or movies based on predefined algorithms and patterns that require a minimal user input.} \cite{Freiknecht:2017:SPG}
This TFG will be focused on terrain generation on a geological scale, disregarding, for the sake of simplicity, the procedural generation of other minor elements of the landscape. 
However, it might prove useful to take the methods used in other areas of PCG into account, such as the prominent use of grammars and L-Systems in products like \textit{SpeedTree} \cite{SpeedTree}, meant for plant generation, or in building generation \cite{Müller:2006:PMB}.

For concrete implementations of terrain generation there exists a varied assortment of methods used. They will be divided roughly as proposed by Valencia-Rosado and Starostenko \cite{Valencia-Rosado:2019:MPT}, condensed in some aspects and expanded in others.

\begin{itemize}
    \item \textbf{Stochastic Methods}:
    The approaches in this block seek to emulate natural randomness through the definition and parametrization of mathematical models.
    These methods tend to be fast and cost-effective, although they lack manipulability and require a deep understanding of the underlying theoretical model to be applied correctly.
    They can be subdivided as follows:
    \begin{itemize}
        \item \textbf{Fractal Methods}:
        Fractals are shapes that exhibit self-similarity on different scales \cite{Mandelbrot:1983:FGN}.
        Most elements of nature can be understood as fractals and, as such, methods for their generation are of great use when attempting to imitate real geography.
        Different methods of fractal generation have been proposed and refined throughout the years, such as \textit{Midpoint Displacement} \cite{Fournier:1982:CRS} \cite{Prusinkiewicz:1993:FMM}, consisting in the subdivision of a mesh's triangle into four smaller triangles, the vertical displacement of the three newly created vertices, and the recursive repetition of these two steps. There is also \textit{Fractal Noise}, mainly popularized by \textit{Perlin Noise} \cite{Perlin:1985:IS}, although other methods exist \cite{Gustavson:2005:SND}, which consists of the combination of multiple layers (or \textit{octaves}) of coherent noise at increasing frequencies and decreasing amplitudes, resulting in a continuous, natural-looking pattern that exhibits statistical self-similarity.
        
        \item \textbf{Grammar Methods}:
        Grammars are sets of rules that define a formal language. 
        They have mostly been used for other types of procedural generation, although some attempts have been made to incorporate them into terrain generation.
    \end{itemize}
    \item \textbf{Simulation Methods}:
    These approaches attempt to emulate the physical processes that form real terrain and, as such, tend to be computationally expensive.
    \begin{itemize}
        \item \textbf{Erosion Simulation}:
        These methods model natural erosion processes such as hydraulic, thermal, wind, or particle-based erosion to shape the landscape \cite{Musgrave:1989:SRE} 
        Modern variants employ multi-scale or GPU-accelerated approaches for real-time applications 
        
        \item \textbf{Procedural Hydrology}:
        Focused on the formation of the river network and drainage basins, these methods simulate water flow and sediment transport across elevation maps \cite{Tzathas:2024:PAE}. 
        Incorporating flow routing improves realism in large-scale terrain structures.
        
        \item \textbf{Plate Tectonics and Thermal Uplift}:
        Macro-scale geological formation can be emulated using plate motion and crust deformation models \cite{Cordonnier:2016:LST}
        These are computationally demanding but generate globally coherent terrain features.
    \end{itemize}
    \item \textbf{Software agents}:Proposed by Doran and Parberry \cite{Doran:2010:CPT} to solve the unnatural looking results fractal noise can produce and to provide more extensive parametrization.
    They implement different \textit{agents}, essentially algorithms, that specifically tackle concrete steps of terrain generation, such as coastline definition, mountain creation or river erosion. 
    This approach seems to greatly facilitate concrete parametrization of different terrain features.
    \item \textbf{Trained Models}:
    These methods train models using mostly real-world terrain data and try to emulate it.
    They have become increasingly popular in the last decade, thanks to the swift developments in the field of artificial intelligence and neural networks, although not many outstanding results have been shown.
\end{itemize}

When considering terrain generation, one shall not limit oneself solely to published scholarly papers on the matter.
As Freiknech and Effelsberg \cite{Freiknecht:2017:SPG} point out in the conclusions to their survey, there exists a considerable disconnect between the advances of scholarly circles and the final implementations of PCG in market-ready games.
Game companies tend to treat their algorithms in an hermetical way and game-centric conferences such as GDC (Game Developer's Conference) usually deal with less technical and more abstract aspects of game development, although some interesting implementations have been presented. \footnote{https://gdcvault.com/play/1024265/Continuous-World-Generation-in-No} \footnote{https://gdcvault.com/play/1025192/Math-for-Game-Programmers-Discrete}
As such, it is reasonable to take a look into the results of PCG in commercially successful games. 
While one may not find a specific procedure to replicate, comparing results can provide valuable insights, which is something most scientific literature is sorely lacking.
The following section will briefly list some interesting case studies and the achievements they showcase that this work aims to build upon.

\begin{itemize}
    \item \textbf{The Elder Scrolls II: Daggerfall} \footnote{https://elderscrolls.bethesda.net/en/daggerfall} is a role playing game (RPG) published in 1996 in which the entire game world, which is around 229,848 square kilometers big, is procedurally generated.
    The game takes places in the fictional provinces of Daggerfall and Hammerfell and sports lore-accurate biomes, elevations and general terrain data.
    An endeavor to update the game for modern devices has been run by the Daggerfall Unity \footnote{https://www.dfworkshop.net/} open source project.
    \item \textbf{Spore} \footnote{https://www.spore.com/} is a videogame published in 2008 in which most content is completely procedurally generated. 
    The player can explore hundreds of planets in a galaxy, each with their own characteristics.
    Flora and fauna are also procedurally generated.
    \item \textbf{Minecraft} \footnote{https://www.minecraft.net} is one of the most commercially successful video games of all time. 
    It procedurally generates its voxel-based maps, creating an expansive world that includes forests, mountains, caves, and oceans. 
    The game's terrain generation algorithm uses a combination of noise functions, such as Perlin noise and simplex noise, to create natural-looking landscapes.
    Players can explore an almost infinite world with unique biomes and features.
    Minecraft's procedural generation also excels in its underground cave systems and resource distribution, which are key to the game's loop. % buscar alguna fuente decente para esto
    \item \textbf{Vintage Story} \footnote{https://www.vintagestory.at} is an independent game that finds its origins in Minecraft modding.
    It exists with the goal of creating a voxel-based world with a more realistic terrain when compared to Minecraft.
    \item \textbf{No Man's Sky} \footnote{https://www.nomanssky.com} is a space exploration game in which most content is completely procedurally generated. 
    The player can explore a theoretical endless number of planets, each with their own characteristics.
    Flora and fauna are also procedurally generated.
\end{itemize}
% \begin{itemize}
%     \item Esta de 2022 \cite{Zhang:2022:SPC} es una de las revisiones más recientes sobre el estado del arte en PCG (Procedural Content Generation).
%     Yo creo que debieras empezar por revisar este.


    
%     \item Maleki \& Zhao (2024) \cite{Maleki:2024:PCG} hacen una revisión reciente sobre generación procedural de contenido en videojuegos.
%     Creo que se centran en el uso de LLMs.

%     \item Este trabajo \cite{Latif:2022:CEP} (de 2022) presenta una revisión crítica de herramientas y algoritmos para generación procedural de terrenos.

%     \item Este \cite{Hendrikx:2013:PCG} (año 2013) es un Survey general sobre generación procedural de contenido en juegos.

%     \item Este otro \cite{Raffe:2012:SPT} (2012) es otro survey, en este caso específico sobre generación procedural de terreno con algoritmos evolutivos.

%     \item Otra revisión más \cite{Smelik:2014:SPM} (2014) sobre generación procedural de mundos virtuales.
% \end{itemize}

\section{Level of detail}
Procedural terrain systems generate vast landscapes, but rendering terrain of such scales in real time requires careful management.
Level of Detail (LOD) techniques seek to reduce rendering costs of assets while preserving perceptual quality, by adjusting the geometric and visual complexity of terrain as a function of different metrics, such as the viewer's position, orientation and the object's importance within the scene.
LOD is thus vital for the feasability of interactive applications including large scale terrains.
The core objective of LOD is to minimize workload without introducing visible artifacts.
Effective LOD systems balance silhouette preservation, continuity and performance, ensuring that downgraded representations do not reveal obvious inconsistencies in the geometry.
When approaching terrain LOD compared to generic LOD, there are some benefits and some difficulties to be found.
The terrain's geometry will generally be more constrained, provide generally uniform height values and the algorithms designed to tackle the LOD can be tailor-made. 
However the large scale of terrain makes it so the geometry can be simultaneously close and far away and might require slower disk paging. \footnote{https://graphics.pixar.com/library/LOD2002/4-terrain.pdf}


Many different LOD systems have been implemented over the years, with different constraints and objectives in mind.\cite{Dalei:2022:RLB}
Each one of these approaches encompasses many different algorithms, they are just a rough division.
\begin{itemize}
    \item \textbf{Discrete LOD (DLOD)}:
    Terrain is statically partitioned into chunks or patches, each with a predefined levels of geometric detail.
    The engine then switches between models as the observer moves.
    This approach is straightforward to implement, benefits greatly from GPU parallelization and has seen widespread usage, although artifacts and seems between patches are a common occurrence, as well as popping.
    
    \item \textbf{Continuous LOD (CLOD)}:
    Dynamic approach that represents the model in an efficient data structure, i.e. quadtrees\cite{Suarez:2015:ETL}, which allows for the desired LOD to be extracted at runtime.
    The on demand generation of LODs helps provide better granularity and smoother transitions.
    
    \item \textbf{View Dependent LOD (VLOD)}:
    A variant of CLOD which takes more viewer details into account.
\end{itemize}

\section{Quality evaluation}
Possessing the capability of evaluating the quality of the generated terrain as well as that of the algorithms used is important.
Over the years there have been many attempts to find universal metrics for the quality and realism of a terrain, although none are too satisfying.
The development of a completely satisfactory universal terrain quality evaluation method is a complex endeavour that lands outside of the scope of this work, which will simply enumerate and apply some already proposed evaluation metrics.

Olsen\cite{Olsen:2004:RPT} proposes Erosion Score, which values low average slope, but high standard deviation. 
This tends to generate interesting terrain with abundant flat areas, but existing height differences.
They later propose the design of specific metrics tailored to the purpose of the generated terrain, in their case a real time strategy computer game.
Rajasekaran et al.\cite{Rajasekaran:2019:PTR} propose a geomorphon-based\cite{Jasiewicz:2013:GPR} statistical analysis validated by the conduction of perceptual studies on humans.


% ---------------------------------------------------
% Trabajo Final de Grado
% Author: Eric Ríos Hamilton <alu0101549835@ull.edu.es>
% Chapter: Results 
% ----------------------------------------------------

\cleardoublepage
\chapter{Case Study: Recognition of road and urban infrastructures in Tenerife} \label{chap:Results}
Con este título u otro, este debiera ser el capítulo "central" del trabajo.

This chapter is a case study for the automatic recognition of roads and urban infrastructures in Tenerife using Sentinel-2 satellite imagery and deep learning architectures. The study evaluates six state-of-the-art segmentation models, analyzing both their performance metrics and computational efficiency in a real-world scenario representative of the Canary Islands' diverse geographic conditions.
Figure \ref{fig:efficiency_analysis_tesla} shows the same analysis performed on a Tesla V100 GPU, which represents a significant improvement in computational capabilities. The Tesla V100 demonstrates substantially faster training times, with most architectures completing training in under 20 minutes, representing a 40-50\% reduction compared to the Quadro M4000. The inference performance also shows notable improvements, with inference times consistently below 15 milliseconds. This GPU shows a balance between training efficiency and inference speed.

% ---------------------------------------------------
% Trabajo Final de Grado
% Author: Eric Ríos Hamilton <alu0101549835@ull.edu.es>
% Chapter: Conclusiones y Líneas de Trabajo Futuras
% ----------------------------------------------------

\chapter{Conclusiones y Líneas de Trabajo Futuras} \label{chap:Conclusiones} 
Traducir a español el correspondiente capítulo en inglés

% ---------------------------------------------------
% Trabajo Final de Grado
% Author: Eric Ríos Hamilton <alu0101549835@ull.edu.es>
% Chapter: Conclusions and Future Lines of Work
% ----------------------------------------------------

\chapter{Conclusions and Future Lines of Work} \label{chap:Conclusions} 
Normalmente este capítulo lo escribiremos al final incluyendo al menos dos apartados: 


\begin{itemize}
    \item Un resumen de todo lo que se ha realizado, investigado y conseguido
    \item Unas conclusiones en sí (científicas) de lo que se puede extraer, concluir, del trabajo realizado
\end{itemize}

\subsection{Future Work}
While the results obtained are promising, several avenues for future work have been identified:

\begin{itemize}
    \item \textbf{Expand to more complex segmentation tasks}, such as multi-class ...

    \item \textbf{Explore the use of more advanced and complex architectures} to improve...

    \item bla, bla, bla...
    \item bla, bla, bla...
\end{itemize}

These directions aim to build upon the foundation established in this work and move toward more robust, scalable, and generalizable remote sensing solutions powered by deep learning.
% ---------------------------------------------------
% Trabajo Final de Grado
% Author: Eric Ríos Hamilton <alu0101549835@ull.edu.es>
% Chapter: Budget
% ----------------------------------------------------

\chapter{Budget} \label{chap:Budget} 
Para elaborar este capítulo, consultar en el RIULL diferentes memorias de TFG que nos parezcan razonables y puedan servir de guía.

This chapter presents a detailed budget for the development and deployment of the research project \textbf{"Application of Deep Learning Techniques in Multispectral Image Processing."}

The budget has been carefully designed to reflect the scope and complexity of a project focused on the implementation and evaluation of six distinct neural network architectures for detecting roads and urban infrastructures in satellite imagery. 
The cost estimation considers both human resources and computational infrastructure, ensuring a realistic projection of the resources required.

The main cost components are:

\begin{itemize}
  \item \textbf{Working hours}: including research, data preprocessing, model development, experimentation, and documentation.
  \item \textbf{Computational resources}: through cloud platforms or local hardware acquisition.
\end{itemize}

\section{Working Hours}
The labor cost is calculated at \textbf{20.00€/hour}, based on the average salary of a junior machine learning/deep learning developer in Spain (35,000€-37,000€ annually), sourced from PayScale\footnote{https://www.payscale.com/research/ES/Job=Machine\_Learning\_Engineer/Salary}, Glassdoor España\footnote{https://www.glassdoor.es/Sueldos/machine-learning-engineer-sueldo-SRCH\_KO0,25.htm}, and specialized tech salary reports from Manfred\footnote{https://www.getmanfred.com/en/blog/guia-salarial-2024-salarios-en-tecnologia-espana-manfred}.

The project spans \textbf{18 weeks} (4.5 months), with \textbf{37.5 hours per week}. Table \ref{tab:work_distribution} shows the distribution of the work.

\begin{table}[H]
\centering
\caption{Estimated cost of working hours by task}
\label{tab:work_distribution}
\begin{tabular}{|l|r|r|}
\hline
\textbf{Task} & \textbf{Hours} & \textbf{Cost (€)} \\
\hline
Literature Review \& Theoretical Framework & 112 & 2,240.00 \\
Data Acquisition \& Preprocessing & 99 & 1,980.00 \\
Architecture Implementation & 225 & 4,500.00 \\
Experimentation \& Evaluation & 183 & 3,660.00 \\
Documentation \& Writing & 56 & 1,120.00 \\
\hline
\textbf{Total} & \textbf{675} & \textbf{13,500.00} \\
\hline
\end{tabular}
\end{table}

\section{Execution Platform}
Two approaches were considered: purchasing hardware or using cloud services.

\subsection*{Hardware Acquisition (Alternative Option)}

\begin{itemize}
  \item \textbf{HPC Node (VERODE Cluster)}:
    \begin{itemize}
      \item \textbf{CPU:} 2× Intel Xeon Gold 6230N @ 2.30 GHz (40 cores).
      \item \textbf{RAM:} 256 GB DDR4 ECC.
      \item \textbf{GPU:} NVIDIA Tesla V100 PCIe (32 GB HBM2).
      \item \textbf{Estimated hardware cost:} 14,700€ – 19,600€.
    \end{itemize}
    \item This configuration matches the computational node used from the VERODE HPC cluster, hosted by the ULL High-Performance Computing Group.
    \item Access to this infrastructure was free of charge for this research project.
\end{itemize}

Based on market listings, the NVIDIA Tesla V100 32GB GPU ranges between \$11,458–\$19,200 USD \cite{microway2022,v100hp}, while the Intel Xeon Gold 6230N processor is listed between \$1,709–\$4,000 USD per unit \cite{intelRCP,distXeon}.

\subsection*{Cloud Computing (Used Option)}

\begin{table}[H]
\centering
\caption{GPU platform costs and specifications used in the project}
\label{tab:gpu_costs}
\begin{tabular}{|l|l|c|c|c|}
\hline
\textbf{GPU Platform} & \textbf{Provider} & \textbf{Hours} & \textbf{Rate/h} & \textbf{Cost (€)} \\
\hline
Quadro M4000 & Paperspace & 120 & 0.45€ & 54.00€ \\
Tesla V100 & Paperspace & 150 & 2.30€ & 345.00€ \\
H100 & Paperspace & 80 & 5.95€ & 476.00€ \\
\hline
\multicolumn{4}{|r|}{\textbf{Total GPU usage cost:}} & \textbf{875.00€} \\
\hline
\multicolumn{4}{|l|}{Pro subscription (\$8)} & $\approx$ 7.44€ \\
\multicolumn{4}{|l|}{Disk storage 1000 GB (\$55)} & $\approx$ 51.15€ \\
\multicolumn{4}{|l|}{Disk storage 100 GB (\$7)} & $\approx$ 6.51€ \\
\multicolumn{4}{|l|}{Custom template fee (\$3.50)} & $\approx$ 3.26€ \\
\hline
\multicolumn{4}{|r|}{\textbf{Estimated monthly cost:}} & \textbf{68.36€} \\
\hline
\multicolumn{4}{|r|}{\textbf{Estimated cost for 18 weeks (4.5 months):}} & \textbf{307.62€} \\
\hline
\multicolumn{4}{|r|}{\textbf{Estimated total cost:}} & \textbf{1,182.62€} \\
\hline
\end{tabular}
\end{table}

Table \ref{tab:gpu_costs} summarizes the computational costs associated with GPU usage during the project, based on the Paperspace platform rates and actual hours used.

In addition to the GPU hourly rates, there are recurring monthly fees for the Paperspace Pro subscription, disk storage, and custom template usage. These overhead costs, estimated at approximately 68.36€ per month, have been computed for the 18-week project duration (about 4.5 months), resulting in an additional cost of 307.62€.

Combining the GPU usage cost (875.00€) with these overhead expenses, the total estimated cloud computing cost for the project amounts to approximately 1,182.62€.

\section{Total Budget Breakdown}

\begin{table}[H]
\centering
\caption{Total budget breakdown by category}
\begin{tabular}{|l|r|r|}
\hline
\textbf{Category} & \textbf{Cost (€)} & \textbf{Percentage} \\
\hline
Working Hours & 13,500.00 & 91.9\% \\
Execution Platform & 1,182.62 & 8.1\% \\
\hline
\textbf{Total} & \textbf{14,682.62} & \textbf{100\%} \\
\hline
\end{tabular}
\end{table}

\section{Final Budget Summary}

\begin{table}[H]
\centering
\caption{Final budget summary by project phase}
\label{tab:final_budget}
\begin{tabular}{|l|r|r|r|}
\hline
\textbf{Phase} & \textbf{Working Hours (€)} & \textbf{Execution (€)} & \textbf{Total (€)} \\
\hline
Research & 2,240.00 & – & 2,240.00 \\
Data Processing & 1,980.00 & – & 1,980.00 \\
Implementation & 4,500.00 & – & 4,500.00 \\
Experimentation & 3,660.00 & 1,182.62 & 4,842.62 \\
Documentation & 1,120.00 & – & 1,120.00 \\
\hline
\textbf{Total} & \textbf{13,500.00} & \textbf{1,182.62} & \textbf{14,682.62} \\
\hline
\end{tabular}
\end{table}

Finally, as shown on Table \ref{tab:final_budget}, the total estimated budget for the project amounts to \textbf{14,682.62€}, covering both human resources and computational infrastructure. This amount represents the cost of carrying out the study presented in this project. Based on this estimate, it is possible to extrapolate the approximate cost of deploying a more ambitious version of the project, such as detecting all roads and urban infrastructure in a larger area beyond Tenerife.


\addcontentsline{toc}{chapter}{Bibliography}
% \bibliographystyle{plain}
\bibliographystyle{ieeetr}
% \bibliographystyle{bmc_article} 
\renewcommand{\bibname}{Bibliography}   %  Para que no aparezca Índice de figuras
\bibliography{bibliography}

%%%%%%%%%%%%%%%%%%%%%%%%%%%%%%%%%%%%%%%%%%%%%%%%%%%%%%%%%%%%%%%%%%%%%%%%%%%%%%%
 
\end{document}
